\begin{appendix}
    \chapter*{Appendices}
    \label{cha:appendices}
    \addcontentsline{toc}{chapter}{Appendices}

    % \includeappendixpdfwithtitle{Task Description from Norkart}{appendices/project_description.pdf}{app:task-description}

    \chapter{Test Results}
    \label{app:test-results}
    \begin{longtable}{lp{1.8cm}p{1.8cm}p{1.8cm}p{1.8cm}p{1.8cm}}
    \caption{Results from GIS benchmark experiment} \label{tbl:test-results-quantitative}                                          \\
    \toprule
    \textbf{Query} \textbf{ID} & \textbf{Agent} \textbf{Type} & \textbf{Outcome} & \textbf{Latency} [\textbf{s}] & \textbf{Tokens} \\
    \midrule
    \endfirsthead
    \caption[]{Results from GIS benchmark experiment}                                                                              \\
    \toprule
    \textbf{Query} \textbf{ID} & \textbf{Agent} \textbf{Type} & \textbf{Outcome} & \textbf{Latency} [\textbf{s}] & \textbf{Tokens} \\
    \midrule
    \endhead
    \midrule
    \multicolumn{5}{r}{Continued on next page}                                                                                     \\
    \midrule
    \endfoot
    \bottomrule
    \endlastfoot
    aker\_brygge\_national     & oaf                          & partial success  & 87.16                         & 3347            \\
    aker\_brygge\_national     & oaf                          & partial success  & 97.85                         & 3042            \\
    aker\_brygge\_national     & oaf                          & partial success  & 70.36                         & 4563            \\
    aker\_brygge\_national     & python                       & partial success  & 68.10                         & 4736            \\
    aker\_brygge\_national     & python                       & partial success  & 49.48                         & 2192            \\
    aker\_brygge\_national     & python                       & partial success  & 76.84                         & 4270            \\
    aker\_brygge\_national     & sql                          & success          & 63.51                         & 2569            \\
    aker\_brygge\_national     & sql                          & success          & 151.46                        & 6273            \\
    aker\_brygge\_national     & sql                          & success          & 103.98                        & 5169            \\
    cliff\_clusters            & oaf                          & partial success  & 93.21                         & 3047            \\
    cliff\_clusters            & oaf                          & partial success  & 134.62                        & 3713            \\
    cliff\_clusters            & oaf                          & failure          & 253.77                        & 6956            \\
    cliff\_clusters            & python                       & failure          & 66.42                         & 5421            \\
    cliff\_clusters            & python                       & failure          & 125.96                        & 4650            \\
    cliff\_clusters            & python                       & failure          & 109.29                        & 4768            \\
    cliff\_clusters            & sql                          & failure          & 39.36                         & 1633            \\
    cliff\_clusters            & sql                          & failure          & 75.27                         & 3085            \\
    cliff\_clusters            & sql                          & success          & 31.64                         & 1632            \\
    county\_names              & oaf                          & success          & 78.08                         & 2389            \\
    county\_names              & oaf                          & success          & 67.45                         & 2399            \\
    county\_names              & oaf                          & failure          & 20.23                         & 1471            \\
    county\_names              & python                       & success          & 44.33                         & 2532            \\
    county\_names              & python                       & success          & 30.50                         & 2355            \\
    county\_names              & python                       & success          & 31.09                         & 2355            \\
    county\_names              & sql                          & success          & 64.21                         & 1874            \\
    county\_names              & sql                          & success          & 52.28                         & 1880            \\
    county\_names              & sql                          & success          & 47.62                         & 1886            \\
    glomma\_counties           & oaf                          & success          & 69.73                         & 2253            \\
    glomma\_counties           & oaf                          & success          & 65.57                         & 1945            \\
    glomma\_counties           & oaf                          & success          & 67.37                         & 2397            \\
    glomma\_counties           & python                       & success          & 663.22                        & 3658            \\
    glomma\_counties           & python                       & failure          & 295.93                        & 4388            \\
    glomma\_counties           & python                       & success          & 285.23                        & 2281            \\
    glomma\_counties           & sql                          & success          & 102.48                        & 1828            \\
    glomma\_counties           & sql                          & success          & 25.01                         & 1266            \\
    glomma\_counties           & sql                          & success          & 21.36                         & 1064            \\
    largest\_county            & oaf                          & success          & 41.82                         & 1713            \\
    largest\_county            & oaf                          & success          & 71.82                         & 2223            \\
    largest\_county            & oaf                          & failure          & 47.00                         & 1797            \\
    largest\_county            & python                       & failure          & 29.37                         & 2415            \\
    largest\_county            & python                       & failure          & 41.95                         & 2810            \\
    largest\_county            & python                       & failure          & 40.55                         & 2407            \\
    largest\_county            & sql                          & success          & 40.08                         & 1771            \\
    largest\_county            & sql                          & success          & 28.51                         & 1474            \\
    largest\_county            & sql                          & success          & 38.30                         & 1381            \\
    nidarosdomen\_polygon      & oaf                          & success          & 29.82                         & 1923            \\
    nidarosdomen\_polygon      & oaf                          & success          & 31.73                         & 1909            \\
    nidarosdomen\_polygon      & oaf                          & success          & 31.61                         & 1922            \\
    nidarosdomen\_polygon      & python                       & success          & 35.10                         & 4510            \\
    nidarosdomen\_polygon      & python                       & success          & 25.36                         & 3649            \\
    nidarosdomen\_polygon      & python                       & success          & 33.35                         & 3663            \\
    nidarosdomen\_polygon      & sql                          & success          & 29.64                         & 1546            \\
    nidarosdomen\_polygon      & sql                          & success          & 90.61                         & 1529            \\
    nidarosdomen\_polygon      & sql                          & success          & 39.39                         & 1513            \\
    num\_trees\_munkegata      & oaf                          & failure          & 75.60                         & 2977            \\
    num\_trees\_munkegata      & oaf                          & failure          & 80.13                         & 2984            \\
    num\_trees\_munkegata      & oaf                          & failure          & 57.66                         & 2407            \\
    num\_trees\_munkegata      & python                       & partial success  & 122.99                        & 3930            \\
    num\_trees\_munkegata      & python                       & failure          & 916.13                        & 2362            \\
    num\_trees\_munkegata      & python                       & failure          & 911.18                        & 8385            \\
    num\_trees\_munkegata      & sql                          & failure          & 80.41                         & 1359            \\
    num\_trees\_munkegata      & sql                          & failure          & 67.41                         & 2626            \\
    num\_trees\_munkegata      & sql                          & failure          & 31.86                         & 1359            \\
    oslo\_bergen\_geodesic     & oaf                          & failure          & 652.88                        & 9253            \\
    oslo\_bergen\_geodesic     & oaf                          & failure          & 102.38                        & 3360            \\
    oslo\_bergen\_geodesic     & oaf                          & failure          & 258.52                        & 6271            \\
    oslo\_bergen\_geodesic     & python                       & partial success  & 118.54                        & 6683            \\
    oslo\_bergen\_geodesic     & python                       & partial success  & 129.40                        & 7782            \\
    oslo\_bergen\_geodesic     & python                       & failure          & 373.37                        & 10412           \\
    oslo\_bergen\_geodesic     & sql                          & failure          & 97.79                         & 2710            \\
    oslo\_bergen\_geodesic     & sql                          & failure          & 73.78                         & 2575            \\
    oslo\_bergen\_geodesic     & sql                          & failure          & 127.43                        & 3318            \\
    oslo\_residental\_diff     & oaf                          & failure          & 119.39                        & 2373            \\
    oslo\_residental\_diff     & oaf                          & partial success  & 114.92                        & 3033            \\
    oslo\_residental\_diff     & oaf                          & success          & 121.17                        & 2621            \\
    oslo\_residental\_diff     & python                       & success          & 673.96                        & 4013            \\
    oslo\_residental\_diff     & python                       & failure          & 2190.08                       & 3867            \\
    oslo\_residental\_diff     & python                       & failure          & 2290.09                       & 3650            \\
    oslo\_residental\_diff     & sql                          & success          & 42.49                         & 1537            \\
    oslo\_residental\_diff     & sql                          & failure          & 94.35                         & 2485            \\
    oslo\_residental\_diff     & sql                          & success          & 52.14                         & 1790            \\
    oslo\_roads\_gte\_70\_kmh  & oaf                          & partial success  & 44.44                         & 1959            \\
    oslo\_roads\_gte\_70\_kmh  & oaf                          & failure          & 58.61                         & 1989            \\
    oslo\_roads\_gte\_70\_kmh  & oaf                          & partial success  & 35.55                         & 1961            \\
    oslo\_roads\_gte\_70\_kmh  & python                       & failure          & 983.76                        & 3912            \\
    oslo\_roads\_gte\_70\_kmh  & python                       & failure          & 757.70                        & 3819            \\
    oslo\_roads\_gte\_70\_kmh  & python                       & failure          & 849.06                        & 3827            \\
    oslo\_roads\_gte\_70\_kmh  & sql                          & success          & 43.88                         & 1542            \\
    oslo\_roads\_gte\_70\_kmh  & sql                          & success          & 37.26                         & 1577            \\
    oslo\_roads\_gte\_70\_kmh  & sql                          & success          & 38.38                         & 1566            \\
    vestfold\_bbox             & oaf                          & failure          & 56.33                         & 2001            \\
    vestfold\_bbox             & oaf                          & partial success  & 50.38                         & 3177            \\
    vestfold\_bbox             & oaf                          & success          & 55.82                         & 2480            \\
    vestfold\_bbox             & python                       & success          & 24.08                         & 1612            \\
    vestfold\_bbox             & python                       & success          & 33.63                         & 2366            \\
    vestfold\_bbox             & python                       & success          & 42.36                         & 2808            \\
    vestfold\_bbox             & sql                          & success          & 23.76                         & 1188            \\
    vestfold\_bbox             & sql                          & success          & 47.35                         & 1932            \\
    vestfold\_bbox             & sql                          & success          & 23.30                         & 1182            \\
    viken\_dissolve            & oaf                          & success          & 52.24                         & 2336            \\
    viken\_dissolve            & oaf                          & success          & 204.07                        & 5078            \\
    viken\_dissolve            & oaf                          & failure          & 133.40                        & 3985            \\
    viken\_dissolve            & python                       & success          & 68.23                         & 6099            \\
    viken\_dissolve            & python                       & success          & 86.32                         & 6007            \\
    viken\_dissolve            & python                       & partial success  & 89.98                         & 6461            \\
    viken\_dissolve            & sql                          & partial success  & 34.58                         & 1553            \\
    viken\_dissolve            & sql                          & success          & 34.72                         & 1540            \\
    viken\_dissolve            & sql                          & partial success  & 37.36                         & 1543            \\
\end{longtable}


    \chapter{Code}

    \section{Python Example}

    \begin{lstlisting}[
            language=Python,
            caption=Python example,
            label=code:python-example
        ]
        import numpy as np
        
        def incmatrix(genl1,genl2):
        m = len(genl1)
        n = len(genl2)
        M = None #to become the incidence matrix
        VT = np.zeros((n*m,1), int)  #dummy variable
        
        #compute the bitwise xor matrix
        M1 = bitxormatrix(genl1)
        M2 = np.triu(bitxormatrix(genl2),1)
        
        for i in range(m-1):
        for j in range(i+1, m):
        [r,c] = np.where(M2 == M1[i,j])
        for k in range(len(r)):
        VT[(i)*n + r[k]] = 1;
        VT[(i)*n + c[k]] = 1;
        VT[(j)*n + r[k]] = 1;
        VT[(j)*n + c[k]] = 1;
        
        if M is None:
        M = np.copy(VT)
        else:
        M = np.concatenate((M, VT), 1)
        
        VT = np.zeros((n*m,1), int)
        
        return M
        
        import numpy as np
        
        def incmatrix(genl1,genl2):
        m = len(genl1)
        n = len(genl2)
        M = None #to become the incidence matrix
        VT = np.zeros((n*m,1), int)  #dummy variable
        
        #compute the bitwise xor matrix
        M1 = bitxormatrix(genl1)
        M2 = np.triu(bitxormatrix(genl2),1)
        
        for i in range(m-1):
        for j in range(i+1, m):
        [r,c] = np.where(M2 == M1[i,j])
        for k in range(len(r)):
        VT[(i)*n + r[k]] = 1;
        VT[(i)*n + c[k]] = 1;
        VT[(j)*n + r[k]] = 1;
        VT[(j)*n + c[k]] = 1;
        
        if M is None:
        M = np.copy(VT)
        else:
        M = np.concatenate((M, VT), 1)
        
        VT = np.zeros((n*m,1), int)
        
        return M


        import numpy as np
        
        def incmatrix(genl1,genl2):
        m = len(genl1)
        n = len(genl2)
        M = None #to become the incidence matrix
        VT = np.zeros((n*m,1), int)  #dummy variable
        
        #compute the bitwise xor matrix
        M1 = bitxormatrix(genl1)
        M2 = np.triu(bitxormatrix(genl2),1)
        
        for i in range(m-1):
        for j in range(i+1, m):
        [r,c] = np.where(M2 == M1[i,j])
        for k in range(len(r)):
        VT[(i)*n + r[k]] = 1;
        VT[(i)*n + c[k]] = 1;
        VT[(j)*n + r[k]] = 1;
        VT[(j)*n + c[k]] = 1;
        
        if M is None:
        M = np.copy(VT)
        else:
        M = np.concatenate((M, VT), 1)
        
        VT = np.zeros((n*m,1), int)
        
        return M
        
        import numpy as np

        ya = "hei"
        
        def incmatrix(genl1,genl2):
        m = len(genl1)
        n = len(genl2)
        M = None #to become the incidence matrix
        VT = np.zeros((n*m,1), int)  #dummy variable
        
        #compute the bitwise xor matrix
        M1 = bitxormatrix(genl1)
        M2 = np.triu(bitxormatrix(genl2),1)
        
        for i in range(m-1):
            for j in range(i+1, m):
                [r,c] = np.where(M2 == M1[i,j])
        for k in range(len(r)):
            VT[(i)*n + r[k]] = 1;
            VT[(i)*n + c[k]] = 1;
            VT[(j)*n + r[k]] = 1;
            VT[(j)*n + c[k]] = 1;
        
        if M is None:
        M = np.copy(VT)
        else:
        M = np.concatenate((M, VT), 1)
        
        VT = np.zeros((n*m,1), int)
        
        return M
        \end{lstlisting}
\end{appendix}