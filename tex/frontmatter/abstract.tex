\section*{Abstract}

\begin{comment}
This paper provides a template for writing a Master's Thesis
(parts of it can also be used when writing a Specialisation Project Report).
The template does not form a compulsory style that you are obliged to use, but rather provides a common starting point for all students. For a given thesis, tuning of the template may still be required, depending on the nature of the thesis and the author's writing style.
Such tuning might involve moving a chapter to a section or vice versa, or removing or adding sections and chapters.

    [If you write a Specialisation Project Report, it should normally focus on the background, related work (i.e., your literature study), and future work sections ---
        with the ``future work'' section containing the plan for the Master's Thesis work to be carried out in the second semester.
        Architectural and experimental sections can also be included, but in preliminary versions.
        All those sections should of course be updated in the Master's Thesis and adapted to the actual work carried out.]

Note that the template contains a lot of examples of how to write different parts of the thesis
as well as how to cite authors and how to use LaTeX and BibTeX.
Some of those examples might only be clear if you actually look at the LaTeX source itself.

The abstract is your sales pitch which encourages people to read your work,
but unlike sales it should be realistic with respect to the contributions of the work.
It should include:
\begin{itemize}
    \item the field of research,
    \item a brief motivation for the work,
    \item what the research topic is,
    \item the research approach(es) applied, and
    \item contributions.
\end{itemize}

The abstract length should be roughly half a page of text (and not more than one page).
It will normally be longer than the abstracts you see in research papers, since some more background / motivation is included.
Do not include lists, tables or figures.
Avoid abbreviations and references.

When writing the abstract, keep in mind that most people might only read this text (and many only the title), so be sure to make it sound good.
What you really want to accomplish is that people who read the abstract will get drawn into your project and read the rest of the text too.
However, the old saying most definitely applies here: You never get a second chance to make a first impression.
\end{comment}

Recent technological developments in \acrlong{acr:nlp} have led to the emergence of powerful \glspl{acr:llm} like those powering ChatGPT --- an \acrshort{acr:ai}-based chat interface created by OpenAI. Such \textit{generative} \glspl{acr:llm} have shown to be versatile, and the overarching goal for this master's thesis is to utilize the logical reasoning and code generation abilities of modern \glspl{acr:llm} to develop a chat-based \acrshort{acr:gis} that can solve \acrshort{acr:gis} tasks based only on the user's natural language problem formulation. The application, named \textit{GeoGPT}, features three different types of agent which can perform common \acrshort{acr:gis} analyses on \gls{acr:osm} data with little to no help from the user, as demonstrated through experiments. GeoGPT relies on the \acrshort{acr:llm}'s \textit{function calling} abilities, which effectively give its agents the ability to use external tools. The agents have different sets of tools, and differ also in the way that they access the \gls{acr:osm} data. One agent accesses the data through a PostGIS database, another through an \acrshort{acr:ogc} \acrshort{acr:api} Features server to download GeoJSON over \acrshort{acr:http}, and the third has direct access to shapefiles stored locally in GeoGPT's environment. Experimental results obtained using a new \acrshort{acr:gis} benchmark show that the \acrshort{acr:sql}/PostGIS agent performs best out of the three, getting the most tasks correct while also being the fastest and cheapest. Results from an experiment conducted to evaluate the importance of the quality of the initial prompt/question from the user show that a more detailed prompt formulated as a step-by-step recipe of solving the \acrshort{acr:gis} task, significantly improves GeoGPT's chances of producing a successful answer. Overall, the work done in this thesis shows that an \acrshort{acr:llm}-based \acrshort{acr:gis} like GeoGPT can solve a variety of common \acrshort{acr:gis} tasks based on simple natural language prompts, but that the user's \acrshort{acr:gis} expertise becomes increasingly important as tasks become more difficult.


\glsresetall
