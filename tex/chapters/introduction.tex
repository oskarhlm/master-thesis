\chapter{Introduction}
\label{cha:introduction}

\begin{comment}
All chapters should begin with an introduction before any sections, giving an overview of the chapter content.
Each section should in addition start with an introduction before its subsections begin.
Chapters with just one section --- or sections with just one sub-section --- should be avoided.
Think carefully about chapter and section titles as each title stands alone in the table of contents (without associated text)
and should convey the meaning of the contents of the chapter or section.

In all chapters and sections it is important to write clearly and concisely. Avoid repetitions and if needed refer back to the original discussion or presentation.
Each new section, subsection or paragraph should provide the reader with new information and be written in your own words. Avoid direct quotes.
If you use direct quotes, unless the quote itself is very significant, you are conveying to the reader that you are unable to express this discussion or fact yourself.
Such direct quotes also break the flow of the language (yours to someone else's).
\end{comment}

The introductory chapter open with explaining the motivation behind the thesis, its goals, and the research questions it will attempt to answer in sections \autoref{sec:background-and-motivation} through \autoref{sec:goals-and-research-questions}. Following this, \autoref{sec:research-method} will explain the research method of this master's thesis, which is to be considered as a technical thesis. \Autosectionref{sec:intro-contributions} will list the main contributions of the thesis, before \autoref{sec:thesis-structure} gives the reader a high-level overview of the thesis to conclude the \nameref{cha:introduction} chapter.

\section{Background and Motivation}
\label{sec:background-and-motivation}

\begin{comment}
Having a template to work from provides a starting point.
However, for a given project, a slight variation in the template may be required due to the nature of the given project.
Furthermore, the order in which the various chapters and sections will be written will also vary from project to project,
but the writing will seldom start at the abstract and sequentially follow the chapters of the report.
One critical reason for this is that you need to start writing as early as possible and that you will begin to write up where you are currently focusing.
However, do not leave working on the abstract until the very last days. The abstract is the first thing anyone reads of an article or thesis --- after the title;
and thus it is important that it is very well written. Abstracts are hard to write, so create revisions throughout the course of your project.

The background and motivation here should state where your project is situated in the field and what the key driving forces motivating this research are.
However, keep this section brief, as it is still part of the introduction.
The motivation will be further elaborated on in Chapter~\ref{cha:related_work}, presenting your complete state-of-the-art.

Note that this template uses italics to highlight where Latin wording is inserted to represent text and the text of the template
that we wish to draw your attention to. The italics themselves are not an indication that such sections should use italics.

\end{comment}

The release of OpenAI's ChatGPT in November, 2022 \citep{openaiIntroducingChatGPT2022} generated a hype within the general population and chat-based systems are now flourishing. Furthermore, significant advancements have been made within code generation, which makes \acrshortpl{acr:llm} useful even for technical tasks, enabling individuals with little to no prior programming experience to carry out computational tasks that require the code execution.

\gls{acr:gis} analysis has traditionally been reserved for \acrshort{acr:gis} \textit{experts}. Furthermore, \acrshort{acr:gis} professional are commonly required to know their way around one or more \glspl{acr:gis}, and to be proficient in programming languages suitable to data science task, such as Python or R. Extensive domain knowledge is often also necessary when tackling \acrshort{acr:gis} tasks, like knowing which data to use for a particular tasks and where to find them. All of these points, and more, are barriers to entry for people that wish to make use of powerful \acrshort{acr:gis} tools for their particular purposes, but lack the technical know-how required to use them correctly. This challenge serves as the overall motivation behind this master's thesis, which will mitigate these issues by utilizing the vast background knowledge and code generation abilities of modern \glspl{acr:llm}.

\section{Goals and Research Questions}
\label{sec:goals-and-research-questions}

\begin{comment}
A research project needs to have one or several question(s) that should be answered.
It is desirable to formulate such questions as early as possible as they provide both an important driving force for the project and clarity as to the goals sought.
However, expect to refine the questions and thus the final path of the project as work progresses.
Any refinements should be conducted with care, so as to avoid that the original aims and previous work are lost.
It is always good to have one (or max two) research goals and perhaps some subgoals,
together with 2--3 explicit research questions (or max four).

\begin{description}
    \item[Goal] \textit{Lorem ipsum dolor sit amet, consectetur adipiscing elit.}
\end{description}

Your goal/objective should be described in a single sentence.
In the text underneath it you can expand on this sentence to clarify what is meant by the short goal description.
The goal of your work is what you are trying to achieve. This can either be the goal of your actual project or
can be a broader goal that you have taken steps towards achieving. Such steps should be expressed in the research questions.
Note that the goal is seldom to build a system. A system is built to enable experiments to be conducted.
The research goal stages the needs that the system is implemented to meet.

\begin{description}
    \item[Research question 1] \textit{Lorem ipsum dolor sit amet, consectetur adipiscing elit.}
\end{description}

Each research question provides a sub-goal and these should be precise and clearly stated enabling the reader to match your results to the original goals.
They will also form the driving force for the experimental plan.

\begin{description}
    \item[Research question 2] \textit{Lorem ipsum dolor sit amet, consectetur adipiscing elit.}
\end{description}

Potentially, how well the goals have been met (and how well the research questions have been answered)
is a theme that you should return to towards the end of the thesis (so in Chapter~\ref{cha:conclusion} and/or Chapter~\ref{cha:discussion}).

For a Specialisation Project, the goal would primarily be to get up to speed with the research field, so the research questions will rather be
limited to exploring what the state-of-the-art is, what methods and data have been used, etc.
A secondary goal of the specialisation is to frame the research questions and goals of the Master's Thesis.
Note that a major difference between the Specialisation Project and the Master's Thesis is that the Master's Thesis work \textit{has\/} to
introduce new research.
Of course the Specialisation Project can also introduce novel work, but there is no such requirement --- and most commonly it does not,
since the core of the project really is to figure out what is ``old'' before you can introduce something which is new.
\end{comment}

Deriving from the motivation described in the section above, the overarching goal of this master's thesis is to investigate the possibilities of utilizing \glspl{acr:llm} to create a natural language interface with a system that is capable of solving \acrshort{acr:gis}-related tasks. The thesis' hypothesis is that modern \glspl{acr:llm} are embedded with an understanding of common \acrshort{acr:gis} workflows, and that their code generation abilities are now of such a level that they can solve a variety of such tasks.

Based on the overarching goal, three research questions have been constructed and are listed below:

\begin{enumerate}
    \item Can an \gls{acr:llm}-based system perform common \acrshort{acr:gis} tasks? \label{rq:gis-question-answering}
    \item What are core challenges in developing \acrshort{acr:llm}-based \acrshortpl{acr:gis}? \label{rq:development-challenges}
    \item Can an autonomous \acrshort{acr:llm}-based \acrshort{acr:gis} agent replace \acrshort{acr:gis} professionals? \label{rq:replaing-gis-professionals}
\end{enumerate}

\section{Research Method}
\label{sec:research-method}

\begin{comment}
What methodology will you apply to address the goals: theoretic/analytic, model/abstraction or design/experiment?
This section will describe the research methodology applied and the reason for this choice of research methodology.
You should return to the actual choices made in the work and the alternatives in the Discussion chapter.
\end{comment}

This master's thesis will be of a technical character, and will revolve around the development of a \enquote{proof of concept}. The usefulness of the \enquote{proof of concept} will be evaluated through a series of tests, that will help answer the research questions listed in the previous section. This report will emphasize the contributions that the \enquote{proof of concept} and the experiments bring, seeing as these occupied a significant majority of the time allocated to the master's thesis.


\section{Contributions}
\label{sec:intro-contributions}

\begin{comment}
This section just provides a brief summary of the main contributions of the work.
The main description of the contributions will come in Section~\ref{sec:contributions}, after the results are presented.
(Hence Section~\ref{sec:introContributions} can also be left out, leaving the discussion completely to Section~\ref{sec:contributions}.)

The format of this section will generally be as follows:

\begin{enumerate}
    \item \textit{Lorem ipsum dolor sit amet, consectetur adipiscing elit.}
    \item \textit{Lorem ipsum dolor sit amet, consectetur adipiscing elit.}
    \item \textit{Lorem ipsum dolor sit amet, consectetur adipiscing elit.}
\end{enumerate}

\noindent
where the items on the list briefly describe the key contributions.

The order of the contributions here is important. List your main contribution first!
Creating this list will help you not only with writing the Conclusion (where all items listed here definitely should be included, and in more detail),
but also with items that need to be mentioned in the Abstract, as well as with points that you will want to bring to attention in the Discussion.
Hence most of the content on this list will be addressed 4--5 times in your text: here, in the Abstract, Discussion, Conclusion, and (most likely)
in the Results chapter.
\end{comment}

Below is a brief description of the contributions of this master's thesis:

\begin{enumerate}
    \item A chat-based \acrshort{acr:gis} named \textit{GeoGPT}, powered by \acrshortpl{acr:llm}, that can solve tasks commonly solved using \acrshort{acr:gis} software.
    \item A new benchmark that will give insigth into the ability of a system like GeoGPT to solve common \acrshort{acr:gis} tasks when provided only with a natural language problem formulation.
    \item An investigation into the open question on the extent to which GeoGPT can replace \acrshort{acr:gis} professionals.
          % \item Experimental results that highlight the importance of the initial problem formulation from the user, suggesting that \acrshort{acr:gis} experience will not become redundant in the foreseeable future.
\end{enumerate}

\section{Thesis Structure}
\label{sec:thesis-structure}

\begin{comment}
This section provides the reader with an overview of what is coming in the next chapters.
You want to say more than what is explicit in the chapter name, if possible, but still keep the description short and to the point. So something along the lines of:

\begin{itemize}
    \item Chapter~\ref{cha:background_theory} introduces the theory, tools and methods necessary to understand the work.
    \item \textit{Lorem ipsum dolor sit amet, consectetur adipiscing elit.}
    \item Chapter~\ref{cha:conclusion} sums up the work and points to ways it can be improved or extended in the future.
\end{itemize}
\end{comment}

Below is an outline of the thesis' structure:

\begin{itemize}
    \item \Autochapterref{cha:background-theory} introduces the theory and tools necessary for the reader to be familiar with in order to understand the rest of the work. It also gives insight into the work that has been done in regard to autonomous, \acrshort{acr:llm}-based systems, in particular those within the field of geomatics.
          % \item \Autochapterref{cha:related_work} gives insight into the work that has been done in regard to autonomous, \acrshort{acr:llm}-based systems, in particular those within the field of geomatics.
          % \item \Autochapterref{cha:data} presents the data provided for GeoGPT in the experiments.
    \item \Autochapterref{cha:architecture} will lay out the GeoGPT's architecture, providing both a high-level overview and details on important parts of the system.
    \item \Autochapterref{cha:experiments} presents the experimental setup, the datasets utilized, and the results and evaluations obtained from these experiments.
    \item \Autochapterref{cha:discussion} will elaborate upon points of discussion that arise from the experimental results.
    \item \Autochapterref{cha:future-work} will suggest areas of improvement that are suitable for future research.
    \item \Autochapterref{cha:conclusion} will conclude the master's thesis, reiterating the main contributions of the work.
\end{itemize}


\glsresetall