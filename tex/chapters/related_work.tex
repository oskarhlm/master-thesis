\chapter{Related Work}
\label{cha:related_work}

\begin{comment}
What other research has been conducted in this area and how is it related to your work?
This section is thus where your literature review will be presented. It is important when presenting the review
that you give an overview of the motivating elements of the work going on in your field and how these relate to your work,
rather than a list of contributors and what they have done.
This means that you need to extract the key important factors for your work and discuss how others have addressed
each of these factors and what the advantages/disadvantages are with such approaches.
As you mention other authors, you should reference their work.
Note that the reference list reflects the literature you have read {\em and\/} have cited.
This will only be a subset of the literature that you have read.

A good way to find relevant work is by checking what others are referencing, e.g., in papers you have already found
or in previous studies carried out at NTNU, such as \cite{Berg;Gopinathan:17}.
However, when doing that,
do not fall into one of the common traps, such as re-iterating someone's false quote or faulty analysis of
a previous paper (check the original source!), or getting stuck inside a local research cluster (a group of
researchers that mainly refer to the ones using the same type of approaches or similar ideas).

Make sure that it is clear how and why you decided to include some references (and discard others). As in all parts of research, it should ideally be possible for someone else to reproduce your work, also when it comes to finding the relevant references.
There are (at least) three basic methods for finding references:
\begin{enumerate}
    \item Trust the authorities (e.g., your supervisor) to dig out good texts for you.
          Those can often be used as a seed set for:
    \item Snowballing, where you have some good articles and check the references in them for other good ones.
          Note that this can be done both backwards and forwards on the timeline; that is, using tools like Google Scholar, you can also check who refers \textit{to\/} the good articles you have already found.
    \item Carry out a Systematic Literature Review (or Structured Literature Review, SLR), a method introduced more formally into Software Engineering by \citet{Kitchenham;Charters:07}, but based on several similar methods for other disciplines.
          At the core of the method is an SQL-related  search over a reference database (such as Google Scholar).
          A good introduction to SLR is given by \citet{Kofod-Petersen:14}.
\end{enumerate}

Note that a reference needs to be complete: you should always give the full name of a conference or journal,
always include page numbers, always say where a book or thesis was published, and where a conference took place, as further described in Section~\ref{sec:reference_list}.

Just as described in the Background chapter (Chapter~\ref{cha:background_theory}), it is possible (and even likely) that you will want to reuse some of the text that you have written for your specialisation project in your Master's Thesis.
This is allowed, as long as it is clearly stated what you have reused and in what form (e.g., if a section is a straight-forward copy, if it has undergone only editorial changes, if it contains some old material but also some new, etc.).
\end{comment}


\section[LLM-based Systems in Geospatial Technologies]{\acrshort{acr:llm}-based Systems in Geospatial Technologies}


\section{Agent Patterns}
\label{sec:agent-patterns}

\subsection{Multi-Agent Pattern}

The multi-agent pattern that takes inspiration from human collaboration in that it is made up from multiple specialized agents that work together to achieve some objective. There have been several implementations of the pattern, with certain differences. MetaGPT \citep{hongMetaGPTMetaProgramming2023} is a \acrshort{acr:llm}-based multi-agent system consisting of agents with human-level domain expertise. Using an assembly line paradigm, where the overall goal is divided into subtasks, \citeauthor{hongMetaGPTMetaProgramming2023} showed that MetaGPT could generate more coherent solutions compared to the previous state-of-the-art multi-agent systems. At the time of release, MetaGPT set a new state-of-the-art performance on the HumanEval and \acrshort{acr:mbpp} benchmarks \citep[7]{hongMetaGPTMetaProgramming2023}, demonstrating the potential of the multi-agent pattern.



\section{LangChain}

LangChain \citep{langchainaiLangchainaiLangchain2022} is an open-source project that provides tooling which simplifies the way developers interface with \glspl{acr:llm}. This tooling includes composable tools and integrations that can be used to build prompts for \acrshortpl{acr:llm}, as well as off-the-shelf chains that perform higher level tasks. Chains are \glspl{acr:dag} --- or sequences of runnables --- that take an input and produces and output. A runnable can be a prompt template with template literals that are substituted with values that are passed into the runnable. The output is the template with the template literals filled in. This output can then be chained into an \acrshort{acr:llm} runnable calls a language model using the prompt template. The output from the \acrshort{acr:llm} runnable could then be passed into an output parser, e.g. a \acrshort{acr:json} parser, that ensures that the chain outputs a \acrshort{acr:json} object. Such chains are the buildings blocks that make up LangChain.

Common use cases for LangChain are:

\begin{itemize}
    \item Building chatbots for question answering that use semantic retrieval from document store
    \item Creating agents with access to external tools be leveraging function calling (see \autoref{sec:function-calling})
    \item Creating code executing agents for Python, \acrshort{acr:sql}, or other programming languages
\end{itemize}

In January 2024, LangChain AI rolled out a new framework called LangGraph which builds on top of the LangChain ecosystem. While the chains commonly found in LangChain are good for \gls{acr:dag} workflow, they are not suited to creating cyclic graphs. LangGraph can be used to add cycles to \acrshort{acr:llm} applications, which are important for agent-like behaviours \citep{langchainaiLangchainaiLanggraph2024}. A graph in LangGraph is a set of nodes that pass some state around, state that can be modified by each node. The nodes are connected together by edges that define what node can succeed another node. These edges can also be conditional, which routes execution to a given node based on the output from a function giving the current state. This allows for complex logic and simplifies implementation of advanced agent patterns, some of which are discussed in \autoref{sec:agent-patterns}.




\glsresetall